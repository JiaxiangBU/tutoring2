\documentclass[]{article}
\usepackage{lmodern}
\usepackage{amssymb,amsmath}
\usepackage{ifxetex,ifluatex}
\usepackage{fixltx2e} % provides \textsubscript
\ifnum 0\ifxetex 1\fi\ifluatex 1\fi=0 % if pdftex
  \usepackage[T1]{fontenc}
  \usepackage[utf8]{inputenc}
\else % if luatex or xelatex
  \ifxetex
    \usepackage{mathspec}
  \else
    \usepackage{fontspec}
  \fi
  \defaultfontfeatures{Ligatures=TeX,Scale=MatchLowercase}
\fi
% use upquote if available, for straight quotes in verbatim environments
\IfFileExists{upquote.sty}{\usepackage{upquote}}{}
% use microtype if available
\IfFileExists{microtype.sty}{%
\usepackage{microtype}
\UseMicrotypeSet[protrusion]{basicmath} % disable protrusion for tt fonts
}{}
\usepackage[margin=1in]{geometry}
\usepackage{hyperref}
\hypersetup{unicode=true,
            pdftitle={【产品】P2P新闻速报},
            pdfborder={0 0 0},
            breaklinks=true}
\urlstyle{same}  % don't use monospace font for urls
\usepackage{graphicx,grffile}
\makeatletter
\def\maxwidth{\ifdim\Gin@nat@width>\linewidth\linewidth\else\Gin@nat@width\fi}
\def\maxheight{\ifdim\Gin@nat@height>\textheight\textheight\else\Gin@nat@height\fi}
\makeatother
% Scale images if necessary, so that they will not overflow the page
% margins by default, and it is still possible to overwrite the defaults
% using explicit options in \includegraphics[width, height, ...]{}
\setkeys{Gin}{width=\maxwidth,height=\maxheight,keepaspectratio}
\IfFileExists{parskip.sty}{%
\usepackage{parskip}
}{% else
\setlength{\parindent}{0pt}
\setlength{\parskip}{6pt plus 2pt minus 1pt}
}
\setlength{\emergencystretch}{3em}  % prevent overfull lines
\providecommand{\tightlist}{%
  \setlength{\itemsep}{0pt}\setlength{\parskip}{0pt}}
\setcounter{secnumdepth}{0}
% Redefines (sub)paragraphs to behave more like sections
\ifx\paragraph\undefined\else
\let\oldparagraph\paragraph
\renewcommand{\paragraph}[1]{\oldparagraph{#1}\mbox{}}
\fi
\ifx\subparagraph\undefined\else
\let\oldsubparagraph\subparagraph
\renewcommand{\subparagraph}[1]{\oldsubparagraph{#1}\mbox{}}
\fi

%%% Use protect on footnotes to avoid problems with footnotes in titles
\let\rmarkdownfootnote\footnote%
\def\footnote{\protect\rmarkdownfootnote}

%%% Change title format to be more compact
\usepackage{titling}

% Create subtitle command for use in maketitle
\newcommand{\subtitle}[1]{
  \posttitle{
    \begin{center}\large#1\end{center}
    }
}

\setlength{\droptitle}{-2em}

  \title{【产品】P2P新闻速报}
    \pretitle{\vspace{\droptitle}\centering\huge}
  \posttitle{\par}
    \author{}
    \preauthor{}\postauthor{}
    \date{}
    \predate{}\postdate{}
  

\begin{document}
\maketitle

{
\setcounter{tocdepth}{2}
\tableofcontents
}
\section{20180910 网贷资产的另一种趋势:引入机构资金}

\href{https://mp.weixin.qq.com/s/5SzFISQapWs4ssCEriwiig}{新流财经}

\textbf{主要观点:目前机构资金的占比升高,资金成本上升。但引入机构资金越来越具有长远战略价值}

对现阶段的P2P平台来说,稳定的资金已经是关乎生死存亡的大事。

7、8月以来的一系列风波,让P2P意识到:除了散户投资者之外,丰富的机构资金来源,可以成为平台构建多元化资金渠道,保持业务稳健的有效策略。

消费金融资产的资金来源本就广泛,银行、保险、信托、ABS都参与了这两年消费金融的爆发,其中最受青睐的当属银行资金。

但P2P背景的资产端,对接银行等传统机构资金并没有那么容易,特别是在今年P2P行业动荡的情况下,简直举步维艰。但即便如此,也有一些P2P资产在对机构融资方面取得了有效进展,以此证明了其资产端的实力。

\textbf{构建机构资金来源}

``银行资金门槛高,成本低,所以有无对接银行资金,是检验资产质量优质与否很好的判断标准'',某P2P系资产端的资金经理易哲告诉新流财经,他每天的工作就是联系大量的城商行、农商行,想方设法找到其中的关键人物,提交项目的计划书。

``有的时候,感觉自己在做无用功,但一旦哪家银行这边有了一点点眉目,就会感觉非常振奋。''

``一般来说能够跟大型银行合作的P2P系资产,要么是行业领头羊,达到IPO水准,要么股东背景非常雄厚。一般的平台根本看不上'',一位消费金融公司CEO向新流表示。

``跟城商行合作的话,主要看机构掌舵人思路活不活,看重的还是利润。''他举例,``比如把钱给到别的消费金融资产,资金成本一般到年化7-9\%,给P2P的话就直接12\%起。同时还收取更高的保证金,一般业务3-5个点,P2P系的起码10个点,甚至还有20个点的。''

即使是难度系数较高,在过去的几年里,仍然也有一些P2P在这方面取得了突破。除了银行资金,不少P2P平台也在对接保险、信托等资金,甚至寻求发行ABS的机会。

比如,2017年,51信用卡20.3\%的资金由机构,包括银行、信托、消费金融公司等机构提供。而乐信集团的资金来源,除了旗下P2P平台桔子理财之外,目前来自银行、机构的资金也占比到一半左右。

根据公开报道,新流财经整理了目前成功对接到传统金融机构资金的P2P平台,不完全名单如下:

\includegraphics{https://note.youdao.com/yws/public/resource/a61462fb5e641f372facb63a5a4404f7/xmlnote/9B243301222B46AD8397CB34E119B9B4/4001}

\begin{figure}
\centering
\includegraphics{fig/xxx/640.jpeg}
\caption{}
\end{figure}

业内人士透露,目前对接P2P资产比较多的银行多为互联网银行和城商行。比较突出的是新网银行、南粤银行、众邦银行、温州银行等,之前还有一些小型的城商行、农商行及区域性银行也对接过P2P资产,比如石嘴山银行等,但目前有一些已经暂停跟P2P的合作。

\section{20180723
催收大幅增加,一季度不良率达2.13\%,浦发信用卡怎么了?}\label{2.13}

\href{https://mp.weixin.qq.com/s/PFAU7Nw6sWgDM2535OQdrw}{新流财经}

\textbf{主要观点:浦发是第一个吃螃蟹的人:线上发卡,获得了巨大的业务扩展。但同时如何平衡是个问题。}

\textbf{浦发坏账增高}:\\
浦发2018年第一季度不良贷款率,比去年年底下降了0.01\%,但依然高达2.13\%。而银行界都知道,爬上2\%的坏账率,在圈内已非常之高。大部分股份银行的不良贷款率,都在1.3\%-1.7\%之间

\textbf{中国发卡量}:\\
央行数据显示,2008年第一季度,中国信用卡发卡量已超过1.47亿张。

\textbf{浦发的方式}:\\
浦发银行的杀手锏,是营销手段与线上获客双管齐下。\\
潘卫东直言:``浦发银行数字化面临的最大挑战是,在数字化条件下加强风险控制。''

\section*{附录}
\addcontentsline{toc}{section}{附录}

这里列举用到实现这个文档的语法。

\begin{enumerate}
\def\labelenumi{\arabic{enumi}.}
\tightlist
\item
  \texttt{\#}、\texttt{\#\#}、\texttt{\#\#\#}分别对应第一、第二、第三级标题,

  \begin{enumerate}
  \def\labelenumii{\arabic{enumii}.}
  \tightlist
  \item
    如实现\texttt{\#\ 20180910\ 网贷资产的另一种趋势:引入机构资金}
  \end{enumerate}
\item
  \texttt{{[}{]}()}用于插入链接,

  \begin{enumerate}
  \def\labelenumii{\arabic{enumii}.}
  \tightlist
  \item
    如实现\texttt{{[}新流财经{]}(https://mp.weixin.qq.com/s/5SzFISQapWs4ssCEriwiig)},这样比较美观,而不应该把raw的链接贴上去。
  \end{enumerate}
\item
  \texttt{**xxx**}用于加粗

  \begin{enumerate}
  \def\labelenumii{\arabic{enumii}.}
  \tightlist
  \item
    如实现\texttt{**主要观点:目前机构资金的占比升高,资金成本上升。但引入机构资金越来越具有长远战略价值**}
  \end{enumerate}
\item
  \texttt{!{[}{]}()}用于插入图片,

  \begin{enumerate}
  \def\labelenumii{\arabic{enumii}.}
  \tightlist
  \item
    如实现\texttt{!{[}640.jpeg{]}(https://note.youdao.com/yws/public/resource/a61462fb5e641f372facb63a5a4404f7/xmlnote/9B243301222B46AD8397CB34E119B9B4/4001)}

    \begin{enumerate}
    \def\labelenumiii{\arabic{enumiii}.}
    \tightlist
    \item
      括号内是图片名字,小括号内是raw链接
    \end{enumerate}
  \end{enumerate}
\item
  在文档前方\texttt{yaml}设置

  \begin{enumerate}
  \def\labelenumii{\arabic{enumii}.}
  \tightlist
  \item
    \texttt{toc:\ yes} 插入目录
  \item
    \texttt{number\_sections:\ yes} 目录排序
  \item
    \texttt{highlight:\ tango} 设置下高亮的模板
  \item
    \texttt{theme:\ cerulean} 设置下主题的模板
  \end{enumerate}
\item
  代码前端插入的\texttt{style},属于css方面的东西,不需要管,为了实现\textbf{行间距为2}
  {[}@hyginn2015{]}
\end{enumerate}

\begin{verbatim}
<style>
p {line-height: 2em;}
</style>
\end{verbatim}

\begin{enumerate}
\def\labelenumi{\arabic{enumi}.}
\tightlist
\item
  实现元素(图片、文字)居中 {[}@小敏纸2015{]}
\end{enumerate}

\begin{verbatim}
<center>
...
</center>
\end{verbatim}


\end{document}
